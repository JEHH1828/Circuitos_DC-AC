\documentclass[10pt]{article}
\usepackage{parskip}
\usepackage[utf8]{inputenc}
\usepackage[left=2.00cm, right=2.00cm, top=2.00cm, bottom=2.00cm]{geometry}
\usepackage[spanish]{babel}
\usepackage{graphicx,subfig}
\usepackage{fancyhdr}
\graphicspath{{Imagenes/}}
\usepackage{enumerate} 
\usepackage{multicol}
\usepackage{tabularx}
\usepackage{amssymb}
\usepackage{adjustbox}
\usepackage{amsmath}
\usepackage{cancel}
\begin{document}


\pagestyle{fancy}
\cfoot{}


%Cabeceras
\rhead{Ley de Ohm.}
\lhead{}

%Portada
\begin{titlepage}
	\newgeometry{
		left=25mm,
		right=25mm,
		top=5mm,
		bottom=30mm,
		headheight = 0 mm
	}

	\begin{figure}[t]
		\subfloat{\includegraphics[width=0.15\textwidth]{Logo_IPN}}
		\hspace{0.6\textwidth}
		\subfloat{\includegraphics[width=0.22\textwidth]{LogoEsime}}
	\end{figure}

	\centering
	{\bfseries\Huge Instituto Politécnico Nacional. \par}
	\vspace{1cm}
	{\scshape\Large Ingeniería en Comunicaciones y Electrónica. \par}
	\vspace{0.3cm}
	{\scshape\Large Laboratorio de Circuitos de C.A y C.U .  \par}
	\vspace{1cm}
	{\scshape\Huge Resistencias Equivalentes. \par}
	\vspace{1cm}
	{\Large 3CM7\par}
	\vfill
	{\Large Autor: \par}

	{\Large José Emilio Hernández Huerta. \par}

	\vfill
	{\Large Septiembre 2023. \par}

\end{titlepage}

\tableofcontents
\newpage

\section{Resumen.}


\begin{multicols}{2}

\section{Objetivo.}



\section{Introducción.}



\section{Marco teórico.}

\subsection{El multímetro.}


\subsection{Antes de usar un multímetro.}


\end{multicols}

\begin{center}
	\begin{adjustbox}{width=500pt}
		\begin{tabular}{|c|c|c|c|c|c|c|c|c|c|c|c|c|c|}
			\hline
			
			 &  & \multicolumn{3}{ |c| }{Teoria} & \multicolumn{3}{ |c| }{Simulación multisim} & \multicolumn{3}{ |c| }{Simulación Pspice} & \multicolumn{3}{ |c| }{Mediciones en laboratorio} \\
			\hline
			 & Valor de resistores en ohms & \multicolumn{3}{ |c| }{$R_{eq}$}  & \multicolumn{3}{ |c| }{$R_{eq}$} & \multicolumn{3}{ |c| }{$R_{eq}$} & \multicolumn{3}{ |c| }{$R_{eq}$} \\
			\hline
			 &  & Tensión & Corriente & Potencia & Tensión & Corriente & Potencia & Tensión & Corriente & Potencia & Tensión & Corriente & Potencia \\ 
			\hline
			$R_{1}$ & 680 & 1 & 1 & 1 & 1 & 1 & 1 & 1 & 1 & 1 & 1 & 1 & 1  \\
			\hline
			$R_{2}$ & 560 & 1 & 1 & 1 & 1 & 1 & 1 & 1 & 1 & 1 & 1 & 1 & 1   \\
			\hline
			$R_{3}$ & 470 & 1 & 1 & 1 & 1 & 1 & 1 & 1 & 1 & 1 & 1 & 1 & 1  \\
			\hline
			$R_{4}$ & 330 & 1 & 1 & 1 & 1 & 1 & 1 & 1 & 1 & 1 & 1 & 1 & 1 \\
			\hline
			$R_{5}$ & 220 & 1 & 1 & 1 & 1 & 1 & 1 & 1 & 1 & 1 & 1 & 1 & 1   \\
			\hline
			$R_{6}$ & 100 & 1 & 1 & 1 & 1 & 1 & 1 & 1 & 1 & 1 & 1 & 1 & 1   \\
			\hline

		\end{tabular}
	\end{adjustbox}
\end{center}

\begin{multicols}{2}
\subsection{Mediciones de continuidad.}


\subsection{Mediciones de diferencia de potencial eléctrico(vóltmetro).}


\subsection{Mediciones de diferencia de potencial eléctrico de corriente alterna(vóltmetro).}


\subsection{Mediciones de intensidad de corriente directa(amperímetro).}


\section{Conclusiones.}

\subsection*{José Emilio Hernández Huerta.}



\begin{thebibliography}{0}
	\bibitem{citekey}[Bragado, I. M. (2003). Física General.]
	\bibitem{citekey}[Benchimol, D. (c. 2020). Electrónica práctica. USERSHOP.]
	\bibitem{citekey}[Peaktech. (2016). Manual de Usuario Peaktech 2005.]
		
\end{thebibliography}

\end{multicols}

\end{document}
